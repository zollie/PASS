\documentclass{article}
\usepackage{graphics} 
\usepackage{hyperref}
\usepackage{fixltx2e}
\usepackage{amssymb}
\usepackage{tikz}
\usepackage{amsmath}
\usepackage{alltt}
\usepackage{verbatim}

\author{Kevin Zollicoffer}
\title{Logistic Regression\\Assignment 2}
\date{06/29/2014}

% no indents
\setlength\parindent{0pt}

\usepackage{Sweave}
\begin{document}
\maketitle
%\tableofcontents
\Sconcordance{concordance:Lesson2.tex:Lesson2.Rnw:%
1 15 1 1 0 96 1}


\section*{Question 1}
Only the logit can be used to estimate the odds ratio for the model predictors. The loglog does not output a probability. If it did one would be able to deduce odds ration. 

%\subsection*{b}
\section*{Question 2}
If the Odds Ratio Confidence Interval of a predictor includes 1, it can be considered not significant. 

\section*{Question 3}
A person with kids has 2.35 greater odss of having an affair than a person without kids all other things constant. 

\section*{Question 4}
The odds of a very religious person having an affair are .27 times as great than that of an anti-religous person all other things constant. 


\section*{Question 5}
$1/.28 = 3.57$ ~\\

A person who is very religous is 3.57 times as likely to not have an affair when compared to an anti-religous person all other things constant. 

\section*{Question 6}
$log(1.2409) = .22$

\section*{Question 7}
The Odds Ratio for \textit{kids} should stay similar or the same because the male predictor is not signifcant (it's confiden interval includes 1) . 

\section*{Question 8}
There is .1981 decrease in the log odds of having Kyphosis for every increase in the start value of vertebrae level for patients who underwent surgery in this study. 

\section*{Question 9}
It is the value of intercept when all predictors are held to 0.

\section*{Question 10}
$ odds_{x1} = 82/157 = .522293$ ~\\
$ odds_{x0} = 431/825 = .5224242$ ~\\
$ odds_{xb} = .522293/.5224242 = .9997489$ ~\\
$ odds_{ln} = log(.9997489) =  -.0002511315$ ~\\

\begin{Schunk}
\begin{Sinput}
> y <- c(0,0,1,1)
> x <- c(0,1,0,1)
> cnt <- c(825,157,431,82)
> modl <- glm(y~x, weights=cnt, family=binomial)
> summary(modl)
\end{Sinput}
\begin{Soutput}
Call:
glm(formula = y ~ x, family = binomial, weights = cnt)

Deviance Residuals: 
     1       2       3       4  
-26.33  -11.49   30.36   13.25  

Coefficients:
              Estimate Std. Error z value Pr(>|z|)    
(Intercept) -0.6492753  0.0594330 -10.924   <2e-16 ***
x           -0.0002513  0.1486498  -0.002    0.999    
---
Signif. codes:  0 '***' 0.001 '**' 0.01 '*' 0.05 '.' 0.1 ' ' 1

(Dispersion parameter for binomial family taken to be 1)

    Null deviance: 1922.9  on 3  degrees of freedom
Residual deviance: 1922.9  on 2  degrees of freedom
AIC: 1926.9

Number of Fisher Scoring iterations: 4
\end{Soutput}
\end{Schunk}


\end{document}
