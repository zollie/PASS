\documentclass{article}
\usepackage{graphics} 
\usepackage{hyperref}
\usepackage{fixltx2e}
\usepackage{amssymb}
\usepackage{tikz}
\usepackage{amsmath}
\usepackage{alltt}
\usepackage{verbatim}

\author{Kevin Zollicoffer}
\title{Logistic Regression\\Assignment 1}
\date{06/22/2014}

% no indents
\setlength\parindent{0pt}

\usepackage{Sweave}
\begin{document}
\maketitle
%\tableofcontents
\Sconcordance{concordance:Lesson1.tex:Lesson1.Rnw:%
1 15 1 1 0 79 1}


\section*{Question 1}
$ log(\frac{\mu}{1-\mu})$

%\subsection*{b}
\section*{Question 2}
The Gaussian link function is an identity function, $g(\mu)$, where g maps it's input back to the same input. The Gaussian link function can handle negative numbers. It's distrbution is Normal, or Gaussian. ~\\ 

In the logit case, a natural log can not be negative (neither can $\frac{\mu}{1-\mu}$). It has a a Binomial distribution. ~\\

\section*{Question 3}
$(82/157) / (431/825) = 1.00$

\section*{Question 4}
$xb_{white} = .25268$ ~\\
$xb_{los} = -.02998$ ~\\
$xb_{con} = -.59868$ ~\\

$odds = (xb_{white})(0)+(xb_{con})+(xb_{los})(10)+(xb_{con})$ ~\\
$odds = (.25268)(0)+(-.59868)+(-.02998)(10)+(-.59868)$ ~\\
$odds = (-.59868)+(-0.89848)$ ~\\
$odds = -1.49716$ ~\\

$p = 1/(1+exp(-xb))$ ~\\
$p = 1/(1+exp(1.49716))$ ~\\
$p = 1/(1+4.468979)$ ~\\
$p = 1/5.468979$ ~\\
$p = .18$ ~\\


\section*{Question 5}

$a3 = sqrt(1.59199)$ ~\\
$a3 = 1.26$ ~\\

\section*{Question 6}

$CI_{95} = \beta+/-1.96*SE(\beta)$ ~\\
$CI_{95l} = -3.344039-1.96*1.26$ ~\\
$CI_{95l} = -5.81$ ~\\

$CI_{95u} = -3.344039+1.96*1.26$ ~\\
$CI_{95u} = -.87$ ~\\

$CI_{95} = [-5.81:-.87]$ ~\\

\section*{Question 7}
Yes, because the interval does not cross $0$. ~\\

\section*{Question 8}

$odds_f = exp(-2.31*0 + 1.00436) = exp(1.00436) = 2.73$ ~\\

\section*{Question 9}
Let $\mu$ be the probability of success of some event $y$ ~\\

$\mu = Pr(y==1)$~

The liklihood or odds of this event occuring is the ratio of success to failure, or ~\\

$ (\frac{\mu}{1-\mu})$

\section*{Question 10}
Given a matrix of form ~\\

$ m = 
\begin{bmatrix}
  A & B \\
  C & D
\end{bmatrix},
$ ~\\

$odds = \frac{AD}{BC}$ ~\\

$risk = \frac{AD + CD}{BC + CD}$ ~\\

The table as given in the question is: ~\\

$ m = 
\begin{bmatrix}
  4 & 8 \\
  3 & 5
\end{bmatrix}
$ ~\\

However the odds and risk ratios for $x$, not $y$ were asked for so we transpose this matrix ~\\

$m^{T} = 
\begin{bmatrix}
  4 & 3 \\
  8 & 5
\end{bmatrix}
$ ~\\

Therefore ~\\

$odds = \frac{4*5}{8*3} = \frac{20}{24} = .83 $ ~\\

$risk = \frac{4*5+8*5}{3*8+3*5} = \frac{60}{39} = 1.54 $ ~\\

\end{document}
